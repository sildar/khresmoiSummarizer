\documentclass[a4paper,10pt]{article}
%\usepackage{libertine}
\usepackage[utf8]{inputenc}

\usepackage[english]{babel}
\usepackage[autolanguage]{numprint}
\usepackage{amsmath}
\usepackage{xcolor}
\usepackage{graphicx}
\usepackage{hyperref}
\usepackage{float}
\graphicspath{{./img/}}
\DeclareGraphicsExtensions{.png}
\renewcommand*\thesection{\arabic{section}}
\hypersetup{
    colorlinks,
    citecolor=blue,
    filecolor=blue,
    linkcolor=blue,
    urlcolor=blue
}

\usepackage{geometry}
%\geometry{scale=0.82, nohead}
\usepackage{listings}
\lstset{keywordstyle=\color{blue}}
\lstset{stringstyle=\color{brown}}
\lstset{showspaces=false}
\lstset{showtabs=false}
\lstset{extendedchars=true}
\lstset{columns=flexible}
\lstset{keepspaces=true}
\lstset{numbers=left, numberstyle=\tiny, stepnumber=1, numbersep=5pt}

\title{CNGL Summarizer adaptation to another language}
\author{ Rémi \textsc{Bois}}


\begin{document}

\maketitle

\tableofcontents

\newpage

\section{Introduction}
\label{sec:intro}

The CNGL Summarizer\footnote{\url{https://github.com/sildar/kreshmoiSummarizer}} is an automated summarizer for medical documents
developped by the CNGL. It is based on sentence extraction and
feature-scoring. It assigns a score to each sentence of a text and
selects the top rated to create a summary. It has been tested on
medical research articles. The original CNGL Summarizer has been
created to summarize English documents. It relies upon English
datasets (list of keywords) to process the document (sentence
recognition, tokenizer, ...) and to score the sentences (eg using the
number of medical terms in the sentence).

This document gives the methods that have been used to adapt this
system to French. It gives not only the French resources that have
been used but also the way they were gathered. Another document has
been created to describe how we created a corpus to test the
summarizer on French documents.

First, we'll describe the English resources used by the
Summarizer. Then we'll explain what have been modified to adapt the
document processing (Tokenization, sentence recognition) to
French. Finally, we'll talk about the resources used to score the
sentences and the way they were gathered.

\section{The English (original) version}
\label{sec:english}

The resources used by the summarizer can be divided into 2 different
categories. First, the document processing resources and second, the
scoring resources. Figure \ref{fig:enresources} gives a list of the
files and the category they belong to.

\begin{figure}[H]
  \centering
  \begin{tabular}[h]{|l|l|c|}
    \hline
    Filename & Use & Category \\
    \hline
    poss\_sent\_end & Indicates which tokens can end a sentence & 1\\
    bad\_sent\_start & Indicates an impossible beginning for a
    sentence & 1\\
    bad\_sent\_end & Indicates an impossible end for a sentence & 1\\
    abbreviations & List of common abbreviations & 1\\
    \hline
    datetime\_terms & Keywords indicating a date or time & 2\\
    sections & Potential title of sections in a paper & 2\\
    cue\_phrases & Keywords indicating that a sentence may be
    important & 2\\
    medical\_dict & Affixes indicating a medical term (eg patho-,
    -trophy) & 2\\
    medical\_terms & List of medical terms & 2\\
    SpacheWordList & List of most common words & 2\\
    stopwords & List of stopwords & 2\\
    \hline
    
  \end{tabular}
  \caption{Resources for the Summarizer}
  \label{fig:enresources}
\end{figure}

The first category includes keywords that help decide where to split
a paragraph into sentences or into words (tokens). Most of these
resources include generic vocabulary and/or punctuation marks. They
are mainly language dependent (as long as we consider languages based
on the Latin alphabet). They can be automatically translated from one
language to another.
The second category is used to score the sentences. Some features
consist in checking if a perticular word is in the processed sentence
(eg is ``In conclusion'' included in the sentence ?). The terms in
these resources are mainly language dependent and often need more that
an automatic translation to give comparable results.


\section{Document processing adaptation}
\label{sec:docproc}

This category includes all resources allowing the Summarizer to split
the paragraphs into sentences and the sentences into words. Theses
resources are language independent and need at most a simple
translation. Some of these files didn't need any modification and the
others had been automatically translated. The figure
\ref{fig:frdocproc} shows how these files where modified for French.

\begin{figure}[H]
  \centering
  \begin{tabular}[h]{|l|l|c|}
    \hline
    Filename & Content & Modification\\
    \hline
    poss\_sent\_end & Punctuation + abbreviations (eg army ranks) &
    None\\
    bad\_sent\_start & Punctuation only & None\\
    bad\_sent\_end & Punctuation only & None\\
    abbreviations & Common abbreviations (eg army ranks) & day and
    titles translation\\
    \hline
  \end{tabular}
  \caption{Resources adaptation for document processing}
  \label{fig:frdocproc}
\end{figure}

The abbreviations list may be adapted or completed with abbreviations
from the medical domain.



\section{Sentence scoring adaptation}
\label{sec:sentscor}

This category includes resources used to score the sentences of a
medical paper in order to choose which ones should be kept as
summary. Most of them couldn't be automatically translated due to the
presence of specialized vocabulary or specific linguistic features
(medical terms, cue phrases). Figure \ref{fig:frsentscor} shows what
modifications have been made on the resources to adapt them to
French. The use of external resources is described section
\ref{sec:extres} page \pageref{sec:extres}.

\begin{figure}[H]
  \centering
  \begin{tabular}[h]{|l|l|c|}
    \hline
    Filename & Content & Modifications \\
    \hline
    datetime\_terms & Keywords indicating a date or time & Automatic
    translation\\
    sections & Potential title of sections in a paper & Manual
    translation\\
    cue\_phrases & Keywords indicating an important sentence & Manual
    translation\\
    medical\_dict & Affixes indicating a medical term (eg patho-) &
    None\\
    medical\_terms & List of medical terms & External resource\\
    SpacheWordList & List of most common words & External resource\\
    stopwords & List of stopwords & External resource\\
    \hline
    
  \end{tabular}
  \caption{Resources adaptation for sentence scoring}
  \label{fig:frsentscor}
\end{figure}

\section{External resources}
\label{sec:extres}

Three resources we used couldn't be automatically translated due to
the need of a particular vocabulary (either simple vocabulary for the
spache words list or a specialized vocabulary for medical terms). We
tried to find multilanguage resources in order to ease the process of
adaptation to another language. Figure \ref{fig:extres} show wich
resources have been used.

\begin{figure}[H]
  \centering
  \begin{description}
  \item[Medical terms] : \hfill \\
    \url{http://users.ugent.be/~rvdstich/eugloss/welcome.html}\\
    List of 1840 (once processed) medical terms, available in 8
    languages, including French and English. This is a work from the
    European Union. It may be outdated (2000) depending on the
    frequence of appearance of new medical terms.\\
    One needs to write a script to extract and normalize the terms of a
    language (get the html pages, then extract the terms for the
    pages).
  \item[Stopwords] : \hfill \\
    \url{http://www.ranks.nl/resources/stopwords.html}\\
    \url{http://snowball.tartarus.org/algorithms/}\\
    Depending on the need of a short or long stopword list, these two
    websites can be used. The first one offers a short list for
    almost 20 languages and the second one offers a long list for a
    dozen languages.
  \item[Spache Word List] : \hfill \\
    An automatic translation may be insufficient due to the probability
    of obtaining specialized words instead of generic easy words. One
    solution is to find a list of words a child should know at a given
    age. For French, we used a list of 1750 words a 4 years old child
    shoud know.\\
    \url{http://fr.wiktionary.org/wiki/Wiktionnaire:Liste_de_1750_mots_fran%C3%A7ais_les_plus_courants}.
      
  \end{description}
  
  \caption{External resources}
  \label{fig:extres}
\end{figure}

\section{Conclusion}
\label{sec:conclusion}

The changes that have been made to adapt the CNGL Summarizer to French
could be easily used to adapt it to other European roman
languages. When an automatic translation wasn't sufficient,
multilingual resources had been used to ease the process. Some manual
translations were needed (in particular the cue phrases translations)
but these are only a few sentences long.



\end{document}