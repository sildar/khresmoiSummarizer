\documentclass[a4paper,10pt]{article}
%\usepackage{libertine}
\usepackage[utf8]{inputenc}

\usepackage[english]{babel}
\usepackage[autolanguage]{numprint}
\usepackage{amsmath}
\usepackage{xcolor}
\usepackage{graphicx}
\usepackage{hyperref}
\usepackage{float}
\graphicspath{{./img/}}
\DeclareGraphicsExtensions{.png}
\renewcommand*\thesection{\arabic{section}}
\hypersetup{
    colorlinks,
    citecolor=blue,
    filecolor=blue,
    linkcolor=blue,
    urlcolor=blue
}

\usepackage{geometry}
%\geometry{scale=0.82, nohead}
\usepackage{listings}
\lstset{keywordstyle=\color{blue}}
\lstset{stringstyle=\color{brown}}
\lstset{showspaces=false}
\lstset{showtabs=false}
\lstset{extendedchars=true}
\lstset{columns=flexible}
\lstset{keepspaces=true}
\lstset{numbers=left, numberstyle=\tiny, stepnumber=1, numbersep=5pt}

\title{Creation of a French medical corpus}
\author{ Rémi \textsc{Bois}}


\begin{document}

\maketitle

\tableofcontents

\newpage

\section{Introduction}
\label{sec:intro}

This document describes the list of processes to follow to create a
medical corpus. It is illustrated by the creation of a French corpus
that have been completed during my internship in the CNGL of
Dublin\footnote{\url{http://www.cngl.ie/}} as part of the porting of a
summariser for the medical domain to French. This internship has been
conducted as part of the
Khresmoi\footnote{\url{http://www.khresmoi.eu/}} project.

Every step, from downloading pdf files to storing the corpus will be
explored. Given the fact that the characteristics of the data
processed depends heavily on the source from where it is extracted,
this document will give a ``generic'' approach. It will nevertheless
provide links where you can find  medical papers in different
languages, scripts that can be adapted to process the files and, most
of all, the method to follow.

Some alternatives will be given, even if they have not been
tested. They might either save time, be easier to use or more adapted to
a specific corpus.

\section{Finding medical papers}
\label{sec:findingpapers}

The first issue in creating a specialized corpus is to find and get
the resources. This section explores some locations where you can find
data and describes the kind of resources you should be looking
for. Finally, basic scripts for downloading a list of pdf files will
be provided.

\subsection{Some locations}
\label{sec:locations}

Finding free medical papers can be a real challenge. Nevertheless,
some websites do aggregate available free papers. While free doesn't
mean that you can distribute them, you can still use them to evaluate
your system.

The website that have been used for the French porting of the
summariser is \url{http://www.freemedicaljournals.com/}. Here, you can
search for papers in a specific language. You'll obtain links to the
journals' websites or their publisher.

\subsection{What kind of papers ?}
\label{sec:kindofpaper}

One of the main issues that you'll face creating a corpus composed of
research articles, is that the articles will likely have different
sections and format. Some will have a list of figures, some will have
multilingual abstracts etc... Since you will have to clean the corpus
in order to make it homogeneous, you want to find papers that follow
the same rules of formatting. This can be achieved by selecting
articles issued from the same journal. By getting articles from a
journal that has been publishing for years, you'll likely have the
same strucure for most of the papers.

Depending on the size of the corpus you are trying to create, you may
have to use different journals. The chances are that you will have to
build custom scripts for cleaning each of the journals, adding a lot
of work for the corpus creation.

Some journals have a ``printable'' version. Those journals are likely
to be composed of only one big pdf file. While attracting at first
sight, remember that you will have to separate the articles if some
happen to be on the same page. This is non-trivial and will not be
explored in this document. The easiest is to find a journal giving a
list of the articles that are published and giving links to download
the pdf files separately.

\subsection{Downloading papers}
\label{sec:download}

Most of the time, the website from which you can download the articles
will be composed of many links, to follow in the right order.

\begin{enumerate}
\item The year of publication
\item The month of publication
\item The articles published in the issue
\end{enumerate}

From there, it is easy to write a script that will explore exery year
in the available range, and to get the html pages containing the list
of articles for each issue.

Once you have those HTML pages, you can download every pdf link that
you find. You may need to use a regular expression (regexp) to only
download articles papers and discard supplements.

On a UNIX system, you will likely use \emph{wget} for downloading
(html pages or pdfs) and \emph{sed} to extract pdf links. You may face
encoding issues which can be resolved thanks to \emph{iconv}.

The script \emph{getpdfs.sh} shows an example with the website
\url{http://www.revuemedecinetropicale.com/html/}. Please note that
this script is very specific and here only to give you an idea of how
to proceed.

\section{From pdf to text}
\label{sec:pdftotext}

Transforming pdf files into text files is almost always a difficult
process. We offer two possibilities here. Depending on the type of
articles you are working with, one may work better than the other.

This step aims at getting all interesting text, without (as much as
possible) getting the footers and headers. Are considered as
non-relevant text :

\begin{itemize}
\item Page numbers
\item Footnotes
\item Journal name and other information that are repeated each page
\end{itemize}

Some information can also be considered as non-important depending on
the experiments you want to run. In this project, we considered as
non-important :

\begin{itemize}
\item The authors' names and their university
\item The references
\item The figures (including their titles)
\item The table of figures / table of contents
\end{itemize}

\subsection{Apache's PDFBox}
\label{sec:apachepdf}

While this tool has not been used in this project, it may definitely
be useful for converting pdf files to text. The
API\footnote{\url{http://pdfbox.apache.org/}} offers to extract the
text page by page. By doing that, you can easily :

\begin{enumerate}
\item Remove footers and headers between each page
\item Join the pages when the text has been split
\item Remove first page information and last pages references
\end{enumerate}

This process should be generic for any given journal.

We do not know if PDFbox handles correctly the multicolumns
articles. If not, this tool will not be usable on most journals.


\subsection{Pdftotext + LApdftext}
\label{sec:lapdf}

This approach has been used in this project. We combined 2 extraction
tools that behave differently to obtain the full text.

LApdftext\footnote{\url{https://github.com/BMKEG/lapdftextProject}} is
a tool allowing to extract text from pdfs based on the characteristics
of the text segments. By writing rules, you can for example decide to
keep all text segments of size 26 but not those which are bold. The
lack of documentation makes this tool hard to use but the default
parameters can give you decent results (that's how we proceeded in
this project). It is still being developped and future versions may be
easier to use.

Pdftotext\footnote{\url{http://linux.die.net/man/1/pdftotext}} is a
unix tool allowing to automatically extract text from a pdf. While
very useful, it does not offer the same amount of personalization as
LApdftext.

In our project, we used pdftotext to extract the title and the
abstract while we used LApdftext for the other text
segments. LApdftext allowed us to not keep the footers and headers and
simplified a lot the cleaning that necessarily comes later. LApdftext
also handles correctly multicolumns, which is very common in research
papers.

The script \emph{combine.py} shows how, from the pdftotext files and
the LApdftext files, we managed to obtain the title, abstract and main
content. The script is not well documented and here only to give you
an idea of how to combine the files. It is not generic (ie, you should
write different scripts for each journal you want to process).

You may want to apply some cleaning before combining the files (eg
solving encoding issues can be mandatory before combining).


\section{Cleaning text}
\label{sec:cleaning}

At the end of this section, the goal is to have clean texts, meaning
that they can be used as a corpus.

\subsection{Checking encoding}
\label{sec:encoding}

The first issue to fix is the encoding. This step can be done as soon
as you have plain text. For doing this, you may need two tools. In
some cases, this step is not necessary.

We insist on the fact that using the UTF-8 standard will allow you to
use your corpus in most environnements while choosing an old or local
encoding (eg latin-1, ascii) will limit you.

The first tool you can use is
iconv\footnote{\url{http://www.gnu.org/savannah-checkouts/gnu/libiconv/documentation/libiconv-1.13/iconv.1.html}},
which can convert from an encoding to another one.

The second tool that may be useful is Fixing Text For
You\footnote{\url{https://www.github.com/LuminosoInsight/python-ftfy}}
(ftfy). This tool has been made to fix encoding issues (like
characters that have been poorly encoded).

\subsection{Removing incoherent files}
\label{sec:incoherent}

Most of the time, some pages of a journal will not be useful. For
example, it is quite common to have an introduction to the journal,
presenting its goal or the company running it. At this step, you
should remove these files.  Most of the time they are easy to detect
automatically (they will not have any abstracts for example). Here are
some of the indicators showing incoherent files.

\begin{itemize}
\item Some common part is missing (abstract, author, title)
\item The title has only one word or contains specific words
\item The main content is too short
\end{itemize}

\subsection{Reconstructing split lines}
\label{sec:splittedlines}

Two columns articles may lead to a lot of words split. You may want to
only have full words and so, remove the hyphens. You have to do this
carefully though because some words naturally contain hyphens (eg
X-ray). The easiest way to solve this problem is to use a
dictionnary. If you find a word containing an hyphen, check if that
word without the hyphen is in the dictionnary. If it is, remove the
hyphen, else, do not touch the word. The larger the dictionnary, the
better the result.

The \emph{rehyphenize.py} script does this. You can feed it with the
dictionnary of your choice, it can work for any roman language. It is
fully documented and we encourage you to use it.


\subsection{Removing useless sections}
\label{sec:removesec}

Due to an imperfect text extraction or simply to some text you want to
strip, you may have to write scripts to remove the useless
sections. This was the case in our project for the abstracts. We
wanted to remove some keywords (eg ``Introduction :'', ``Results :'')
that made the comparison with summaries complicated.

The script \emph{rmsections.py} has been used. It is fully documented
and you can apply it to a custom list of keywords to remove from a
specific line.


\section{Formatting the data for storage}
\label{sec:format}

Once you have plain text files, you have to choose the format you will
use to distribute it. Two major approches exist : plain text or XML.

\subsection{XML vs plain text}
\label{sec:xml}

XML is a widely used format, allowing to type the information and to
modify it easily. If you can, thanks to the previous processing, to
obtain the authors' names, the date of publication, you may consider
formatting your corpus with XML. It will then be easy to add new
informations like the number of citations for each paper or the
name of the conference where is has been presented. These kind of
informations can lead to new use of your corpus.

On the other hand, if you do not plan on using your corpus again, you
may keep it in plain text, as long as the structure is ``simple and
obvious''. This means that specific sections (title, abstract, ...)
should always be on the same line. You may want to add a readme to
explain the structure.

\end{document}