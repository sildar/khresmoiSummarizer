\documentclass[a4paper,10pt]{article}
%\usepackage{libertine}
\usepackage[utf8]{inputenc}

\usepackage[english]{babel}
\usepackage[autolanguage]{numprint}
\usepackage{amsmath}
\usepackage{xcolor}
\usepackage{graphicx}
\usepackage{hyperref}
\usepackage{float}
\graphicspath{{./img/}}
\DeclareGraphicsExtensions{.png}
\renewcommand*\thesection{\arabic{section}}
\hypersetup{
    colorlinks,
    citecolor=blue,
    filecolor=blue,
    linkcolor=blue,
    urlcolor=blue
}

\usepackage{geometry}
%\geometry{scale=0.82, nohead}
\usepackage{listings}
\lstset{keywordstyle=\color{blue}}
\lstset{stringstyle=\color{brown}}
\lstset{showspaces=false}
\lstset{showtabs=false}
\lstset{extendedchars=true}
\lstset{columns=flexible}
\lstset{keepspaces=true}
\lstset{numbers=left, numberstyle=\tiny, stepnumber=1, numbersep=5pt}

\title{Creation of a French medical corpus}
\author{ Rémi \textsc{Bois}}


\begin{document}

\maketitle

\tableofcontents

\newpage

\section{Introduction}
\label{sec:intro}

This document describes the list of processes to follow to create a
medical corpus. It is illustrated by the creation of a French corpus
that have been completed during my internship in the CNGL of
Dublin\footnote{\url{http://www.cngl.ie/}} as part of the porting of a
summariser for the medical domain to French. This internship has been
conducted as part of the
Khresmoi\footnote{\url{http://www.khresmoi.eu/}} project.

Every step, from downloading pdf files to storing the corpus will be
explored. Given the fact that the characteristics of the data
processed depends heavily on the source from where it is extracted,
this document will give a ``generic'' approach. It will nevertheless
provide links where you can find  medical papers in different
languages, scripts that can be adapted to process the files and, most
of all, the method to follow.

Some alternatives will be given, even if they have not been
tested. They might either save time, be easier to use or more adapted to
a specific corpus.

\section{Finding medical papers}
\label{sec:findingpapers}

The first issue in creating a specialized corpus is to find and get
the resources. This section explores some locations where you can find
data and describes the kind of resources you should be looking
for. Finally, basic scripts for downloading a list of pdf files will
be provided.

\subsection{Some locations}
\label{sec:locations}

Finding free medical papers can be a real challenge. Nevertheless,
some websites do aggregate available free papers. While free doesn't
mean that you can distribute them, you can still use them to evaluate
your system.

The website that have been used for the French porting of the
summariser is \url{http://www.freemedicaljournals.com/}. Here, you can
search for papers in a specific language. You'll obtain links to the
journals' websites or their publisher.

\subsection{What kind of papers ?}
\label{sec:kindofpaper}

One of the main issues that you'll face creating a corpus composed of
research articles, is that the articles will likely have different
sections and format. Some will have a list of figures, some will have
multilingual abstracts etc... Since you will have to clean the corpus
in order to make it homogeneous, you want to find papers that follow
the same rules of formatting. This can be achieved by selecting
articles issued from the same journal. By getting articles from a
journal that has been publishing for years, you'll likely have the
same strucure for most of the papers.

Depending on the size of the corpus you are trying to create, you may
have to use different journals. The chances are that you will have to
build custom scripts for cleaning each of the journals, adding a lot
of work for the corpus creation.

Some journals have a ``printable'' version. Those journals are likely
to be composed of only one big pdf file. While attracting at first
sight, remember that you will have to separate the articles if some
happen to be on the same page. This is non-trivial and will not be
explored in this document. The easiest is to find a journal giving a
list of the articles that are published and giving links to download
the pdf files separately.

\subsection{Downloading papers}
\label{sec:download}

Most of the time, the website from which you can download the articles
will be composed of many links, to follow in the right order.

\begin{enumerate}
\item The year of publication
\item The month of publication
\item The articles published in the issue
\end{enumerate}

From there, it is easy to write a script that will explore exery year
in the available range, and to get the html pages containing the list
of articles for each issue.

Once you have those HTML pages, you can download every pdf link that
you find. You may need to use a regular expression (regexp) to only
download articles papers and discard supplements.

On a UNIX system, you will likely use \emph{wget} for downloading
(html pages or pdfs) and \emph{sed} to extract pdf links. You may face
encoding issues which can be resolved thanks to \emph{iconv}.

The script \emph{getpdfs.sh} shows an example with the website
\url{http://www.revuemedecinetropicale.com/html/}. Please note that
this script is very specific and here only to give you an idea of how
to proceed.

\section{From pdf to text}
\label{sec:pdftotext}

\subsection{Apache's pdf to text}
\label{sec:apachepdf}

\subsection{Pdftotext + LApdftext}
\label{sec:lapdf}



\section{Cleaning text}
\label{sec:cleaning}

\subsection{Checking encoding}
\label{sec:encoding}

\subsection{Removing incoherent files}
\label{sec:incoherent}



\subsection{Removing useless sections}
\label{sec:removesec}



\section{Formatting the data for storage}
\label{sec:format}

\subsection{Give a structure to the text}
\label{sec:textstructure}

\subsection{XML vs plain text}
\label{sec:xml}

\section{Conclusion}
\label{sec:conclusion}






\end{document}