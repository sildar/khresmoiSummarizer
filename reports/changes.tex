\documentclass[a4paper,10pt]{article}
%\usepackage{libertine}
\usepackage[utf8]{inputenc}

\usepackage[english]{babel}
\usepackage[autolanguage]{numprint}
\usepackage{amsmath}
\usepackage{xcolor}
\usepackage{graphicx}
\usepackage{hyperref}
\graphicspath{{./img/}}
\DeclareGraphicsExtensions{.png}
\renewcommand*\thesection{\arabic{section}}
\hypersetup{
    colorlinks,
    citecolor=blue,
    filecolor=blue,
    linkcolor=blue,
    urlcolor=blue
}

\usepackage{geometry}
\geometry{scale=0.82, nohead}
\usepackage{listings}
\lstset{keywordstyle=\color{blue}}
\lstset{stringstyle=\color{brown}}
\lstset{showspaces=false}
\lstset{showtabs=false}
\lstset{extendedchars=true}
\lstset{columns=flexible}
\lstset{keepspaces=true}
\lstset{numbers=left, numberstyle=\tiny, stepnumber=1, numbersep=5pt}

\title{Modifications on the CNGL Summarizer}
\author{ Rémi \textsc{Bois}}


\begin{document}

\maketitle

%\tableofcontents

\section{Changes made}
\label{sec:changes}

\begin{description}
\item [Improvement of the readme] : \hfill
  7f27ab05503934785304beb6e1e3963b55dc7465 \\
  Added commands to use the two libraries that are not available via
  the pom.xml (jaws and jwi). \\
  Added the instructions to run the main class.

\item[Resources added] : \hfill
  8587c35e3ef7c5d194993ccc471cd63c7fe399e7 \\
  Added a structure to have multiple languages available (one folder
  for each language). Modified the XML settings to consider the
  language that is set.

\item[Paragrapher Changes] : \hfill
  15e3ee93811e139c96aeb9e9ef9d0ed32296b715 \\
  Readability/Code Logic : The paragrapher is supposed to split a
  String into paragraphs. Previous behavior was that the paragrapher
  also tokenized the the text. The paragrapher now returns an
  $ArrayList<String>$ which can be tokenized via the tokenizer.

\item[Aggregator changes] : \hfill
  15e3ee93811e139c96aeb9e9ef9d0ed32296b715 \\
  Replaced the $SentenceScore[]$ by an $ArrayList$
  ($ArrayList<SentenceScore>$)\\
  Readability : put the score of a sentence to $-\infty$ if it has
  been discarded by a feature.\\
  Simplification : changed the custom list sorting by the
  Collections.sort() native method.\\
  See : b58407f3c86f42e84696a60f1d3253bff2b14786 too

\item[TokenUtils changes] : \hfill
  15e3ee93811e139c96aeb9e9ef9d0ed32296b715 \\
  The method recombineTokens1d was supposed to transform an arraylist
  of tokens to a String. The previous behavior relied on uninitialized
  variables and was replaced by the StringUtils.join() methods which
  has the same goal and is more efficient.

\item[Sentence and Paragraph] : \hfill
  46e5d275a825a44243602a24fbb2674bce06e4af \\
  Replaced the $ArrayList<TokenInfo>$ by a class Sentence.\\
  Replaced the $ArrayList<ArrayList<TokenInfo>>$ by a class Paragraph
  which is an $ArrayList<Sentence>$.\\
  Replaced the PageStructure by an $ArrayList<Paragraph>$ (instead of
  an $ArrayList<ArrayList<ArrayList<TokenInfo>>>$.\\
  See : 3f00b835b15ab1045a552916e8224ccc3f4a76e2 too.

\item[PageStructure] : \hfill 5bb5def7e2017cbfe6696c401f67f7635d5e12e8
  \\
  Replaced the PageStructure by an $ArrayList<Paragraph>$ (instead of
  an $ArrayList<ArrayList<ArrayList<TokenInfo>>>$.\\
  Removed the getStructure method since it now is the PageStructure
  instance itself that can be used.\\
  See : 46e5d275a825a44243602a24fbb2674bce06e4af too.

\item [Efficiency] : \hfill 229adb5260ac1e2bfb0de164be4909d61d647b57
  \\
  The program had to reinitialize every feature when it processed
  several documents in a row. This has been modified and geatly
  increased the performance (from 10s per document to 0.2s per
  document).

\item[Readability / Debug] : \hfill
  0fdde226e600f786276e53ad94691e5cb2d268f1 \\
  Replaced the $ArrayList<Double[]>$ given as a result for the scores
  of all features by a HashMap containing the name of the feature that
  gave the score. This allows to have a file with all the scores for
  each sentence and each feature and helps for debug/feature impact.

\item[Punctuation Feature] : \hfill
  216d7972fbcf4904e7c3cceba9a7c16b9b2d57b3 \\
  Use of the Java Punctuation class ($\backslash p\{P\}$) to recognize
  punctuation. This affected the recognized tokens : before, any
  punctuation mark in a token counted the token as a punctuation. Now,
  only the tokens composed only of punctuation are considered
  punctuation. We may want to increase the ratio (currently 30\%) used
  to discard a sentence.

\item[Section Importance] : \hfill \\
  Added : Abbreviations as a section title, with a negative
  score. Medical papers often include a list of abbreviations that are
  used in the paper. This should not appear in the summary.

\item[Affix and basic words] : \hfill
  d17b91e8546b3abdf60c7c84426c0828d05133ef \\
  Bug fix : these features returned NaN when the sentence was only
  composed of stopwords or punctuations. Since the NaN discarded the
  sentences, -1 has been chosen as a replacement when fixing the
  bug. We may want to choose 0 as score instead, in order to not
  discard them.

\item[Lucene Features] : \hfill
  348bc8d46c292c2435ef5d35b4b93b9263c3669d\\
  Bug Fix : the system wasn't set to handle multiple documents
  processing in a row. The Lucene index was never emptied, leading to
  incoherent results and exceptions. The index is now emptied before
  each time it is built.

\end{description}

\section{Resources used for french}
\label{sec:res4french}

This section gives the location of the resources that have been used.

\begin{description}
\item[Medical terms] : \hfill \\
  \url{http://users.ugent.be/~rvdstich/eugloss/welcome.html}\\
  List of 1840 (once processed) medical terms, available in 8
  languages, including French and English. This is a work from the
  European Union. It may be outdated (2000) depending on the frequence
  of appearance of new medical terms.
  I wrote a script to extract and normalize the terms of a language.
\item[Stopwords] : \hfill \\
  \url{http://www.ranks.nl/stopwords/french.html}\\
  List of 126 french stopwords. It may be too short and be extended by
  \url{http://snowball.tartarus.org/algorithms/french/stop.txt}.
\item[Cue phrases] : \hfill \\
  Simple manual translation.
\item[Spache Word List] : \hfill \\
  An automatic translation by a dictionnary could be enough since
  these are simple words. We could also use a wikitionnaire article :
  \url{http://fr.wiktionary.org/wiki/Wiktionnaire:Liste_de_1750_mots_fran%C3%A7ais_les_plus_courants}.

\item[Abbreviations] : \hfill \\
  Simple translation of the dates. The military ranks, list of states
  weren't changed but don't seem very useful for the medical
  domain. Added Mme and Mlle to the person names section.
\item[Sections] : \hfill \\
  Simple manual translation.
  
    


\end{description}


\section{Changes to make}
\label{sec:todo}

These are propositions on things to improve.

\begin{description}
\item[Cue Phrases] : \hfill \\
  Add : ``In this study'' \\
  Add : ``We'' (as in ``We proved that'', ``We demonstrate that'',
  ...)\\
\item[DateTime] : \hfill \\
  The presence of dates is very limited in the tests I ran on english
  medical papers. This feature only checks the presence of days and
  months. Maybe it shouldn't be used. If used, we have to remove the
  month of May.

\item[Skimming Feature] : \hfill \\
  Seems buggy. The boosting seems to only return 1 for each
  sentence. We may want to use it out of Lucene.

\item[Section Depth] : New Feature \hfill \\
  Maybe the text inside the sections is more important than the text
  on subsubsections. It is similar to the skimming feature.

\item[Citation Presence] : \hfill \\
  To discuss : The absence of citation in a sentence tends to indicate
  original work. On the contrary, citation presence often indicates
  the subject that is discussed in the paper.
\item[Long sentences] : \hfill \\
  We already have a feature discarding sentences if they are too
  short (often discards figures titles, section names, ...). We could
  discard the sentences that are too long and which mainly consist in
  :
  \begin{itemize}
  \item Citations
  \item Experimental protocols
  \item lists (abbreviation lists, authors and lab lists, ...)
  \end{itemize}

\item[Lucene Features] : \hfill \\
  The Skimming feature, the TS-ISF (TF-ISF), and the Named Entities
  feature use Lucene. They compute a boost for some terms (eg
  capitalized words for the named entities), and then run a query via
  Lucene with some extra-work (like stopwords removal and
  normalization). This seems confusing. In the Named Entities feature
  for example, the boost for a sentence is $NamedEntities^2 /
  numberOfTerms$ where the $numberOfTerms$ excludes stopwords,
  punctuation, and other terms. We need to go through every sentence,
  boost the sentences, then build a Lucene index, get the score from
  Lucene when a single sentence analysis could be enough (eg, number
  of capitalized words for the named entities feature).

  The TS-ISF is implemented as a basic Lucene Feature without any
  boosting.

\end{description}


\section{Features Impact}
\label{sec:impact}

To do.



\end{document}
